\documentclass[11pt,letterpaper]{article}
\usepackage[utf8]{inputenc}
%\usepackage[T1]{fontenc}
%\usepackage{microtype}

\usepackage[colorlinks=true,linkcolor=orange,urlcolor=orange,citecolor=blue,anchorcolor=green,pdfusetitle]{hyperref}
\usepackage{amsmath, amssymb, amsthm}
\usepackage{amsfonts}
\usepackage{dsfont}
\usepackage{mathtools}

\usepackage{fullpage}
\usepackage{setspace}
\onehalfspacing

\usepackage{mathpazo}
\usepackage{ifthen}
\usepackage{enumerate}

\usepackage{framed}
\usepackage{mdframed}
\usepackage{tcolorbox}
\tcbuselibrary{breakable}


\setlength{\parindent}{0pt}


\DeclareMathOperator{\tr}{tr}
\newcommand{\cH}{\mathcal{H}}
\newcommand{\cB}{\mathcal{B}}
\newcommand{\cU}{\mathcal{U}}
\newcommand{\cM}{\mathcal{M}}
\newcommand{\cL}{\mathcal{L}}
\newcommand{\bF}{\mathbb{F}}
\newcommand{\bR}{\mathbb{R}}
\newcommand{\cS}{\mathcal{S}}
\newcommand{\one}{\mathds{1}}
\DeclareMathOperator{\opvec}{vec}
\DeclareMathOperator{\rk}{rk}
\title{MATH 416 Homework Week 3}
\author{Jacob Beckey}

\newcommand{\newex}[2]{
	\ifthenelse{\equal{#2}{1}}{\noindent\textbf{Exercise #1} (#2 point):}{\noindent\textbf{Exercise #1} (#2 points):}
}

\begin{document}
\begin{center}
\textbf{\large MATH 416 Abstract Linear Algebra}

\vspace{.5em}Week 10 - Homework 8 \\ \textbf{Assigned:} Fri. Oct. 31, 2025 \\ \textbf{Due:} Fri. Nov. 7, 2025 (by 8pm) \\ 

\end{center}
\textbf{Reminder:} I encourage you to work together and use resources as needed. Please remember to state who you collaborated with and what resources you used. \\

\newex{1}{5} Orthogonal Bases

Suppose $e_1, \dots, e_n$ is an orthonormal basis of $V$. 
\begin{enumerate}
    \item[(a)] (3 points) Prove that if $v_1, \dots, v_n$ are vectors in $V$ such that 
    \begin{align}
        \|e_k - v_k\| < \frac{1}{\sqrt{n}}
    \end{align}
    for each $k$, then $v_1, \dots, v_n$ is a basis of $V$.
    \item[(b)] Show that there exist $v_1, \dots, v_n \in V$ such that 
    \begin{align}
        \|e_k -v_k \| \leq \frac{1}{\sqrt{n}}
    \end{align}
    for each $k,$ but $v_1, \dots, v_n$ is \textit{not} linearly independent.
\end{enumerate}
\textit{Note: The first part of this exercise shows that if we perturb an orthonormal basis an appropriate amount, we still have a basis. The second part shows that we can't increase the $1/\sqrt{n}$.}

\newpage
\bigskip\newex{2}{5} Inner products and orthogonal complements

\begin{enumerate}[(i)]
\item (3 points) Let $U\leq \mathbb{R}^4$ be the subspace spanned by the vectors $v_1 = (1,2,3,-4)^T$ and $v_2 = (-5,4,3,2)^T$.
Find orthonormal bases for $U$ and its orthogonal complement $U^\perp$ for the standard inner product $\langle x,y\rangle = x_1y_1 + x_2 y_2 + x_3 y_3 + x_4 y_4$.
\item (2 points) Consider the following inner product on $\mathbb{R}^3$: $\langle x,y\rangle_{\mathrm{alt}} \coloneqq 2 x_1 y_1 + x_2 y_2 + 2x_3 y_3$.
Compute $\lbrace v\rbrace^\perp$ for the vector $v = (1,-2,1)^T\in\mathbb{R}^3$.

{\itshape Warning: If you use the Gram-Schmidt (GS) procedure for this example, then you need to use the inner product $\langle x,y\rangle_{\mathrm{alt}}$ and the associated norm $\|x\|_{\mathrm{alt}} \coloneqq \sqrt{\langle x,x\rangle_{\mathrm{alt}}}$ in the GS-formulas.}
\end{enumerate}

\newpage
\bigskip \newex{3}{5} Orthogonal projections

\begin{enumerate}[(i)]
    \item (2 points) Suppose $u,v \in V$. Prove that $\langle u,v \rangle = 0\Leftrightarrow \|u\| \leq \|u +a v\|$ for all $a \in \mathbb{F}$.
	\item (1 point) Let $U\leq V$ be a subspace of a finite-dimensional inner product space $V$. Show that $P_{U^\perp} = I_V - P_U$.
    \item (2 points) Suppose $V$ is finite-dimensional and $P \in \mathcal{L}(V)$ is such that $P^2=P$ and $\|Pv\| \leq \|v\|$ for every $v \in V$. Prove that there exists a subspace $U$ of $V$ such that $P=P_U$. 
\end{enumerate}
 
\end{document}