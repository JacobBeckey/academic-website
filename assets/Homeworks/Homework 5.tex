\documentclass[11pt,letterpaper]{article}
\usepackage[utf8]{inputenc}
%\usepackage[T1]{fontenc}
%\usepackage{microtype}

\usepackage[colorlinks=true,linkcolor=blue,urlcolor=blue,citecolor=blue,anchorcolor=green,pdfusetitle]{hyperref}
\usepackage{amsmath, amssymb, amsthm}
\usepackage{amsfonts}
\usepackage{dsfont}
\usepackage{mathtools}

\usepackage{fullpage}
\usepackage{setspace}
\onehalfspacing

\usepackage{mathpazo}
\usepackage{ifthen}
\usepackage{enumerate}

\usepackage{framed}
\usepackage{mdframed}
\usepackage{tcolorbox}
\tcbuselibrary{breakable}


\setlength{\parindent}{0pt}


\DeclareMathOperator{\tr}{tr}
\newcommand{\cH}{\mathcal{H}}
\newcommand{\cB}{\mathcal{B}}
\newcommand{\cU}{\mathcal{U}}
\newcommand{\cM}{\mathcal{M}}
\newcommand{\cL}{\mathcal{L}}
\newcommand{\bF}{\mathbb{F}}
\newcommand{\bR}{\mathbb{R}}
\newcommand{\cS}{\mathcal{S}}
\newcommand{\one}{\mathds{1}}
\DeclareMathOperator{\opvec}{vec}
\DeclareMathOperator{\rk}{rk}
\title{MATH 416 Homework Week 3}
\author{Jacob Beckey}

\newcommand{\newex}[2]{
	\ifthenelse{\equal{#2}{1}}{\noindent\textbf{Exercise #1} (#2 point):}{\noindent\textbf{Exercise #1} (#2 points):}
}

\begin{document}
\begin{center}
\textbf{\large MATH 416 Abstract Linear Algebra}

\vspace{.5em}Week 6 - Homework 5 \\ \textbf{Assigned:} Fri. Oct. 3, 2025 \\ \textbf{Due:} Fri. Oct. 10, 2025 (by 8pm) \\ 

\end{center}
\textbf{Reminder:} I encourage you to work together and use resources as needed. Please remember to state who you collaborated with and what resources you used. \\

\newex{1}{10} \textbf{Rank of a Matrix}\\

Consider the matrix
\[
A = \begin{bmatrix}
1 & 0 & 1 \\
0 & 1 & 1 \\
1 & 1 & 2
\end{bmatrix}.
\]

\begin{enumerate}
\item[(i)] (3 points). Compute the column rank and the row rank of \(A\) by finding maximal linearly independent sets of columns and of rows.

\item[(ii)] (3 points). Prove the following proposition (the \emph{column–row factorization}):  
If the column rank of a matrix \(A \in M_{m \times n}(\mathbb{F})\) is \(r\), then there exist matrices
\[
C \in M_{m \times r}(\mathbb{F}), \qquad R \in M_{r \times n}(\mathbb{F})
\]
such that
\[
A = C R.
\]
\textit{Hint: let the columns of \(C\) be a maximal linearly independent set of columns of \(A\), and argue that every column of \(A\) is a linear combination of these.}
\item[(iii)](1 point). Find $C$ and $R$ such that their product yields the $A$ from part (i).
\item[(iv)] (3 points). Use the factorization in (ii) to prove that the column rank of any matrix equals its row rank.  
\textit{Hint: Take transposes and interpret column rank of \(A^T\) as row rank of \(A\).}
\end{enumerate}




\medskip\newex{2}{10} \textbf{Invertibility, Isomorphisms, and Basis Change}

\begin{enumerate}
    \item[(i)] (2 points). Suppose $T \in \mathcal{L}(U,V)$ and  $S \in \mathcal{L}(V,W)$ are both invertible. Prove that $ST \in \mathcal{L}(U,W)$ is invertible and $(ST)^{-1} = T^{-1} S^{-1}$.
     \item[(ii)] (4 points). Suppose $V$ is finite-dimensional and $S,T \in \mathcal{L}(V).$ Prove that
     \begin{align*}
         ST \text{ is invertible} \Leftrightarrow S,T \text{ are both invertible}.
     \end{align*}
     \item[(iii)] (4 points). Suppose $V$ is finite-dimensional and $T \in \mathcal{L}(V)$. Prove that $T$ has the same matrix with respect to every basis of $V$ if and only if $T$ is a scalar multiple of the identity.
    \end{enumerate}


\textbf{(optional) Bonus Question} (2 points): \textbf{Rank-1 Decomposition}\\

Prove that every matrix $A \in M_{m \times n}(\mathbb{F})$ of rank $r$ can be written as a sum of $r$ matrices of rank $1$. 
That is, show that there exist column vectors $u_1, \dots, u_r \in \mathbb{F}^{m \times 1}$ and row vectors $v_1^T, \dots, v_r^T \in \mathbb{F}^{1 \times n}$ such that
\[
A = u_1 v_1^T + u_2 v_2^T + \cdots + u_r v_r^T.
\]

\textit{Motivation.} This statement shows that we can build up matrices in terms of ``rank-1 outer products'' (i.e. a column times a row vector) building blocks. Later, we will see that the so-called \textit{Singular Value Decomposition (SVD)} is a refinement of this idea: it expresses any matrix as a sum of rank-1 outer products, but with the additional structure that the vectors form orthonormal bases and the coefficients are nonnegative singular values. The SVD has far-reaching consequences for data science, machine learning, entanglement theory, and many more fields. 



\end{document}