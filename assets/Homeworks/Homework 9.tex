\documentclass[11pt,letterpaper]{article}
\usepackage[utf8]{inputenc}
%\usepackage[T1]{fontenc}
%\usepackage{microtype}

\usepackage[colorlinks=true,linkcolor=orange,urlcolor=orange,citecolor=blue,anchorcolor=green,pdfusetitle]{hyperref}
\usepackage{amsmath, amssymb, amsthm}
\usepackage{amsfonts}
\usepackage{dsfont}
\usepackage{mathtools}

\usepackage{fullpage}
\usepackage{setspace}
\onehalfspacing

\usepackage{mathpazo}
\usepackage{ifthen}
\usepackage{enumerate}

\usepackage{framed}
\usepackage{mdframed}
\usepackage{tcolorbox}
\tcbuselibrary{breakable}


\setlength{\parindent}{0pt}


\DeclareMathOperator{\tr}{tr}
\newcommand{\cH}{\mathcal{H}}
\newcommand{\cB}{\mathcal{B}}
\newcommand{\cU}{\mathcal{U}}
\newcommand{\cM}{\mathcal{M}}
\newcommand{\cL}{\mathcal{L}}
\newcommand{\bF}{\mathbb{F}}
\newcommand{\bR}{\mathbb{R}}
\newcommand{\cS}{\mathcal{S}}
\newcommand{\one}{\mathds{1}}
\DeclareMathOperator{\opvec}{vec}
\DeclareMathOperator{\rk}{rk}
\title{MATH 416 Homework Week 3}
\author{Jacob Beckey}

\newcommand{\newex}[2]{
	\ifthenelse{\equal{#2}{1}}{\noindent\textbf{Exercise #1} (#2 point):}{\noindent\textbf{Exercise #1} (#2 points):}
}

\begin{document}
\begin{center}
\textbf{\large MATH 416 Abstract Linear Algebra}

\vspace{.5em}Week 11 - Homework 9 \\ \textbf{Assigned:} Fri. Nov. 7, 2025 \\ \textbf{Due:} Fri. Nov. 14, 2025 (by 8pm) \\ 

\end{center}
\textbf{Reminder:} I encourage you to work together and use resources as needed. Please remember to state who you collaborated with and what resources you used. \\

\newex{1}{5} \textbf{Minimization via Orthogonal Projection}\\
 
Find $p \in \mathcal{P}_3(\mathbb{R})$ such that $p(0)=0, p'(0)=0,$ and $\int_{0}^1 |2+3x-p(x)|^2 dx$ is as small as possible.

\newpage
\bigskip\newex{2}{5} \textbf{Adjoints and Self-Adjoint Operators}\\

\begin{enumerate}[(a)]
    \item (3 points) Suppose $V$ is finite dimensional and $\varphi$ is a linear functional on $V$ (i.e. $\varphi \in \mathcal{L}(V,\mathbb{F})).$ Then, there is a unique vector $v \in V$ such that 
\begin{align}
    \varphi(u) = \langle u ,v \rangle,
\end{align}
for every $u \in V$.
\item (2 points) Use (a) to argue why the definition of the adjoint makes sense. 
\end{enumerate}
\textit{Hint: The result in part (a) is called the Riesz representation theorem and you may find it useful to peruse Axler 6B to learn more!}

\newpage

\bigskip \newex{3}{5} \textbf{Spectral Theorem}

Consider the self-adjoint matrix
\begin{align*}
A = \begin{pmatrix}
2 & 1-i\\
1+i & 3
\end{pmatrix}.
\end{align*}
\begin{enumerate}[(a)]
    \item (2 points) Prove that a normal operator on a complex inner product space is self-adjoint if and only if all its eigenvalues are real.
	\item (2 points) Find the eigenvalues of $A$ and an orthonormal basis $\cB$ for $\mathbb{C}^2$ consisting of eigenvectors.
	\item (1 point) Let $U = \cM(I)_{\cB,\cS}$, and compute $U^*AU$. What do you find?
\end{enumerate}
 \newpage 

 \newpage
\noindent \textbf{(Optional) Bonus Question} (3 points): \textit{Self-adjoint maps and Pauli matrices}\\
Some of the most important objects in theoretical physics are the Pauli matrices $I,X,Y,Z\in M_2(\mathbb{C})$, defined as
\begin{align*}
I &= \begin{pmatrix}
1 & 0\\0&1
\end{pmatrix} &
X&= \begin{pmatrix}
0 & 1\\ 1&0
\end{pmatrix}&
Y&= \begin{pmatrix}
0 & -i\\ i & 0
\end{pmatrix}&
Z&= \begin{pmatrix}
1&0\\0&-1
\end{pmatrix}.
\end{align*}
Let the \textit{real} vector space of all self-adjoint complex $(2\times 2)$-matrices be defined $\cH_2 = \lbrace A\in M_2(\mathbb{C})\colon A^* = A\rbrace$. Moreover, let us define an inner product on this space as
    \begin{align}
        \langle A,B \rangle = \tr{[AB]},
    \end{align}
    where the \textit{trace} of a matrix is defined as $\tr{[A]} = \sum_{i=1}^n A_{ii}$ (i.e. the sum of the diagonal terms). 
\begin{enumerate}[(a)]
    \item (1 point)  Show that $\lbrace I,X,Y,Z\rbrace$ is a linearly independent list with respect to this inner product.
    \item (1 point) Formally prove that the $\dim_{\mathbb{R}} \cH_2 = 4$.
    \item (1 points) What are the eigenvalues of these matrices? 

\end{enumerate}

\end{document}