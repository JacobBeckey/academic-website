\documentclass[11pt,letterpaper]{article}
\usepackage[utf8]{inputenc}
%\usepackage[T1]{fontenc}
%\usepackage{microtype}

\usepackage[colorlinks=true,linkcolor=blue,urlcolor=blue,citecolor=blue,anchorcolor=green,pdfusetitle]{hyperref}
\usepackage{amsmath, amssymb, amsthm}
\usepackage{amsfonts}
\usepackage{dsfont}
\usepackage{mathtools}

\usepackage{fullpage}
\usepackage{setspace}
\onehalfspacing

\usepackage{mathpazo}
\usepackage{ifthen}
\usepackage{enumerate}

\usepackage{framed}
\usepackage{mdframed}
\usepackage{tcolorbox}
\tcbuselibrary{breakable}


\setlength{\parindent}{0pt}


\DeclareMathOperator{\tr}{tr}
\newcommand{\cH}{\mathcal{H}}
\newcommand{\cB}{\mathcal{B}}
\newcommand{\cU}{\mathcal{U}}
\newcommand{\cM}{\mathcal{M}}
\newcommand{\cL}{\mathcal{L}}
\newcommand{\bF}{\mathbb{F}}
\newcommand{\bR}{\mathbb{R}}
\newcommand{\cS}{\mathcal{S}}
\newcommand{\one}{\mathds{1}}
\DeclareMathOperator{\opvec}{vec}
\DeclareMathOperator{\rk}{rk}
\title{MATH 416 Homework Week 3}
\author{Jacob Beckey}

\newcommand{\newex}[2]{
	\ifthenelse{\equal{#2}{1}}{\noindent\textbf{Exercise #1} (#2 point):}{\noindent\textbf{Exercise #1} (#2 points):}
}

\begin{document}
\begin{center}
\textbf{\large MATH 416 Abstract Linear Algebra}

\vspace{.5em}Week 5 - Homework 4 \\ \textbf{Assigned:} Fri. Sept. 26, 2025 \\ \textbf{Due:} Fri. Oct. 3, 2025 (by 8pm) \\ 

\end{center}
\textbf{Reminder:} I encourage you to work together and use resources as needed. Please remember to state who you collaborated with and what resources you used. \\

\newex{1}{6} Injectivity and surjectivity

Let $V$ be a (finite-dim.) vector space over a field $\mathbb{F}$. 
Let $v_1,\dots,v_m\in V$ and define the linear map 
\begin{align*}
T\colon \mathbb{F}^m &\to V,\quad 
\begin{pmatrix}
x_1\\ \vdots\\ x_m
\end{pmatrix} \mapsto \sum_{i=1}^m x_i v_i.
\end{align*}
\begin{enumerate}[(i)]
	\item Prove that $T$ is injective if and only if $\lbrace v_1,\dots,v_m\rbrace$ are linearly independent.
	\item Prove that $T$ is surjective if and only if $V = \mathrm{span}\{ v_1,\dots,v_m\}$.
\end{enumerate}

\medskip\newex{2}{4} Linear maps as matrices I

Let $V,W$ be finite-dimensional vector spaces over a field $\mathbb{F}$, and fix bases $\cB_V = \lbrace v_1,\dots, v_n\rbrace$ for $V$ and $\cB_W = \lbrace w_1,\dots,w_m\rbrace$ for $W$.
In the following, we abbreviate $\cM(\cdot) = \cM(\cdot)_{\cB_V,\cB_W}$.

Show that:

\begin{enumerate}[(i)]
	\item $\cM(S+T) = \cM(S) + \cM(T)$ for $S,T\in\cL_\bF(V,W)$.
	\item $\cM(aT) = a \cM(T)$ for $a\in\bF$ and $T\in\cL_\bF(V,W)$.
\end{enumerate}

\medskip\newex{3}{4} Linear maps as matrices II

Consider the following linear map:
\begin{align*}
T\colon \bR^3 &\to \bR^4,\quad 
\begin{pmatrix}
x_1\\x_2\\x_3
\end{pmatrix}
\mapsto
\begin{pmatrix}
x_1 - x_2 + x_3\\
x_1 - x_2 - x_3\\
x_1 + x_2\\
x_2 - x_3
\end{pmatrix}
\end{align*}
\begin{enumerate}[(i)]
	\item Determine $\cM(T)_{\cS_3,\cS_4}$, where $\cS_3$ and $\cS_4$ are the standard bases in $\bR^3$ and $\bR^4$, respectively.
	\item Let now 
	\begin{align*}
	\cB_V &= \left\lbrace \left(\begin{smallmatrix}
	1\\0\\0
	\end{smallmatrix}\right),
	\left(\begin{smallmatrix}
	1\\1\\0
	\end{smallmatrix}\right)
	,
	\left(\begin{smallmatrix}
	1\\1\\1
	\end{smallmatrix}\right)\right\rbrace & 
	\cB_W &= \left\lbrace \left(\begin{smallmatrix}
	1\\0\\1\\0
	\end{smallmatrix}\right)
	,
	\left(\begin{smallmatrix}
	0\\-1\\1\\0
	\end{smallmatrix}\right)
	,
	\left(\begin{smallmatrix}
	0\\0\\1\\1
	\end{smallmatrix}\right)
	,
	\left(\begin{smallmatrix}
	0\\0\\1\\0
	\end{smallmatrix}\right)\right\rbrace
	\end{align*}
	and determine $\cM(T)_{\cB_V,\cB_W}$.
	
	{\itshape Remark: You do not need to show that $\cB_V$ and $\cB_W$ are indeed bases for $\bR^3$ and $\bR^4$, respectively.}
\end{enumerate}

\textbf{(optional) Bonus Question} (2 points): \textbf{Projection Operators}\\
Let $P \in \mathcal{L}(V)$ such that $P^2 =P$. Such operators are called \textit{projection operators} or \textit{projectors} and they play a fundamental role in linear algebra, representation theory, quantum mechanics, data science, and many more fields. 

\begin{enumerate}
    \item[(i)] Suppose $P \in \mathcal{L}(V)$ and $P^2 = P$. Prove that $V = \operatorname{null}{P} \oplus \operatorname{range}{P}$.
\end{enumerate}


\end{document}