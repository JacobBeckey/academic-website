\documentclass[11pt,letterpaper]{article}
\usepackage[utf8]{inputenc}
%\usepackage[T1]{fontenc}
%\usepackage{microtype}

\usepackage[colorlinks=true,linkcolor=blue,urlcolor=blue,citecolor=blue,anchorcolor=green,pdfusetitle]{hyperref}
\usepackage{amsmath, amssymb, amsthm}
\usepackage{amsfonts}
\usepackage{dsfont}
\usepackage{mathtools}

% Theorem environments
\newtheorem{theorem}{Theorem} 
\newtheorem{lemma}[theorem]{Lemma}       % shares numbering with theorem
\newtheorem{corollary}[theorem]{Corollary}

% For remarks/definitions with different styling:
\theoremstyle{definition}
\newtheorem{definition}[theorem]{Definition}
\theoremstyle{remark}
\newtheorem*{remark}{Remark}

\usepackage{fullpage}
\usepackage{setspace}
\onehalfspacing

\usepackage{mathpazo}
\usepackage{ifthen}


\setlength{\parindent}{0pt}

\DeclareMathOperator{\tr}{tr}
\newcommand{\cH}{\mathcal{H}}
\newcommand{\cB}{\mathcal{B}}
\newcommand{\cU}{\mathcal{U}}
\newcommand{\one}{\mathds{1}}
\DeclareMathOperator{\opvec}{vec}
\DeclareMathOperator{\rk}{rk}

\title{MATH 416 Homework Week 1}
\author{Felix Leditzky}

\newcommand{\newex}[2]{
	\ifthenelse{\equal{#2}{1}}{\noindent\textbf{Exercise #1} (#2 point):}{\noindent\textbf{Exercise #1} (#2 points):}
}

\begin{document}
\begin{center}
\textbf{\large MATH 416 Abstract Linear Algebra}

\vspace{.5em}Homework 1 \\ \textbf{Assigned:} Fri. August 29, 2025 \\ \textbf{Due:} Fri. Sept. 5, 2025 (by 1pm)
\end{center}

\newex{1}{10} In class, we saw one way of taking products between two vectors. In this problem, you will practice your proof writing skills to prove a few fundamental results about the dot product. First, let us formally define the dot product between two real vectors.
\begin{definition}
For vectors $\mathbf{x} = (x_1,\dots,x_n)$ and $\mathbf{y} = (y_1,\dots,y_n)$ in $\mathbb{R}^n$, the \textbf{dot product} is defined as
\begin{align*}
\mathbf{x} \cdot \mathbf{y} &= \sum_{i=1}^n x_i y_i.
\end{align*}
We also define the \textbf{norm} (or length) of a vector as
\begin{align*}
\|\mathbf{x}\| &= \sqrt{\mathbf{x} \cdot \mathbf{x}}.
\end{align*}
\end{definition}
Now, let's warm up with some elementary results.
\begin{enumerate}
    \item[1a.] (1 point) Compute $(1,2,3)\cdot (4,5,6)$.
    \item[1b.] (1 point) Show that $\mathbf{x}\cdot \mathbf{y} = \mathbf{y}\cdot \mathbf{x}$ (commutativity).
    \item[1c.] (1 point) Show that $\mathbf{x}\cdot (\mathbf{y}+\mathbf{z}) = \mathbf{x}\cdot \mathbf{y} + \mathbf{x}\cdot \mathbf{z}$ (distributivity).
    \item[1d.] (1 point) Show that $\mathbf{x}\cdot \mathbf{x} \geq 0$ and equals $0$ if and only if $\mathbf{x}=\mathbf{0}$.
\end{enumerate}
With these results in mind, we now turn to the proof of one of the most ubiquitous inequalities in mathematics, the \textit{Cauchy-Schwarz Inequality}.
\begin{theorem}[Cauchy--Schwarz Inequality]
For all $\mathbf{x},\mathbf{y} \in \mathbb{R}^n$,
\begin{align*}
|\mathbf{x} \cdot \mathbf{y}| \leq \|\mathbf{x}\| \,\|\mathbf{y}\|.
\end{align*}
\end{theorem}
\begin{enumerate}
    \item[1e.] (6 points) Prove the Cauchy-Schwarz inequality. 
\end{enumerate}


\newex{2}{5}
Consider the following system of linear equations:
	\begin{align*}
	a_{11} x_1 + \dots + a_{1n} x_n &= b_1\\
	&\vdotswithin{=}\\
	a_{m1} x_1 + \dots + a_{mn} x_n &= b_m,
	\end{align*}
	where $a_{ij},b_i\in\mathbb{R}$ for $1\leq i\leq m, 1\leq j\leq n$.
	Show that the set of solutions of this system does not change under the following operations:
	\begin{enumerate}
		\item\label{item:mult} \textbf{Multiplication.} Multiply both sides of an equation by $\lambda\neq 0$.
            \item\label{item:perm} \textbf{Permutation.} Swapping any two rows.
		\item\label{item:add} \textbf{Addition.} Add one row to another one.
	\end{enumerate}
	{\itshape Hint: Assume there exists a solution $\mathbf{x} = (x_1,\dots,x_n)^T$ of the original system of linear equations, and show that it is also a solution of the system transformed via the operations in \ref{item:mult}, \ref{item:perm}, and \ref{item:add} }
	
\vspace*{1em}\newex{3}{10}
Determine if the following systems of linear equations have solutions, and if yes find them.
In one of the systems below there are multiple solutions. What is the dimensionality of the solution space? (1 point)
\begin{enumerate}
	\item (3 points) System of linear equations in $\mathbb{R}^3$
	\begin{align*}
	x_1 + 2x_2 + 2 x_3 &= 4 \\
	x_1 + 3 x_2 + 3 x_3 &= 5\\
	2 x_1 + 6x_2 + 5 x_3 &= 6
	\end{align*}
	\item (3 points) System of linear equations in $\mathbb{R}^4$
	\begin{align*}
	x_1 + 2 x_2 + x_4 &= 7\\
	x_1 + x_2 + x_3 - x_4 &= 3\\
	3 x_1 + x_2 + 5 x_3 - 7 x_4 &=1
	\end{align*}
	\item (3 points) System of linear equations in $\mathbb{R}^4$
	\begin{align*}
	2 x_1 + x_2 + 7x_3-7x_4 &= 2\\
	-3 x_1 + 4 x_2 -5x_3-6x_4 &= 3\\
	x_1 + x_2 + 4x_3 -5x_4 &= 2
	\end{align*}
\end{enumerate}


\end{document}